\documentclass[12pt, titlepage]{article}

\usepackage{booktabs}
\usepackage{tabularx}
\usepackage{hyperref}
\hypersetup{
    colorlinks,
    citecolor=blue,
    filecolor=black,
    linkcolor=red,
    urlcolor=blue
}
\usepackage[round]{natbib}
\usepackage{amsmath, mathtools}
\usepackage{amssymb}

%% Comments

\usepackage{color}

\newif\ifcomments\commentstrue

\ifcomments
\newcommand{\authornote}[3]{\textcolor{#1}{[#3 ---#2]}}
\newcommand{\todo}[1]{\textcolor{red}{[TODO: #1]}}
\else
\newcommand{\authornote}[3]{}
\newcommand{\todo}[1]{}
\fi

\newcommand{\wss}[1]{\authornote{blue}{SS}{#1}} 
\newcommand{\plt}[1]{\authornote{magenta}{TPLT}{#1}} %For explanation of the template
\newcommand{\an}[1]{\authornote{cyan}{Author}{#1}}

%% Common Parts

\newcommand{\progname}{ProgName} % PUT YOUR PROGRAM NAME HERE %Every program
                                % should have a name


\begin{document}

\title{Project Title: System Verification and Validation Plan for EOMEE} 
\author{Gabriela S\'anchez D\'iaz}
\date{\today}
	
\maketitle

\pagenumbering{roman}

\section{Revision History}

\begin{tabularx}{\textwidth}{p{3cm}p{2cm}X}
\toprule {\bf Date} & {\bf Version} & {\bf Notes}\\
\midrule
29-10-2020 & 1.0 & Created VnV first draft \\
%Date 2 & 1.1 & Notes\\
\bottomrule
\end{tabularx}

\newpage

\tableofcontents

\listoftables

\listoffigures

\newpage

\section{Symbols, Abbreviations and Acronyms}

\renewcommand{\arraystretch}{1.2}
\begin{tabular}{l l} 
  \toprule		
  \textbf{symbol} & \textbf{description}\\
  \midrule 
  T & Test\\
  VnV & Verification and Validation\\
  DD & Data Definition\\
  EOMEE& Equation-of-motion for excited states\\
  IM & Instance Model\\
  MO & Molecular orbital(s)\\
  HOMO& Highest Occupied Molecular Orbital\\
  LUMO& Lowest Unoccupied Molecular Orbital\\
  R & Requirement\\
  SRS & Software Requirements Specification\\
  RDM & Reduced density matrix\\
  TDM & Transition density matrix\\
  $h_{pq}$ & 1-electron integral\\
  $v_{pqrs}$ & 2-electron integral\\
  $\gamma_{pq}$ & 1-electron reduced density matrix\\
  $\Gamma_{pqrs}$ & 2-electron reduced density matrix\\
  \bottomrule
\end{tabular}\\

%\wss{symbols, abbreviations or acronyms -- you can simply reference the SRS
%  \citep{SRS} tables, if appropriate}

\newpage

\pagenumbering{arabic}

This document presents the VnV plan for EOMEE. First, the strategies for the 
verifycation of the SRS, code desing and implementation will be described, 
followed by the software validation plan. The later one will present the system 
and unit tests that will verify the functional and nonfunctional requirements 
of the package.
%\wss{provide an introductory blurb and roadmap of the  Verification and 
%Validation plan}

\section{General Information}

\subsection{Summary}

%\wss{Say what software is being tested.  Give its name and a brief overview of
%  its general functions.}\\
The present document presents the validation plan of the package Equation of 
Motion for Excited States (EOMEE). This program is intended as a research tool 
for evaluating excited states (including ionizad states) and their related 
spectroscopic properties such as the oscillator strengths. It also strives to 
assess the effect of the electron correlation on these properties.

\subsection{Objectives}

%\wss{State what is intended to be accomplished.  The objective will be around
%  the qualities that are most important for your project.  You might have
%  something like: ``build confidence in the software correctness,''
%  ``demonstrate adequate usability.'' etc.  You won't list all of the 
%qualities,
%  just those that are most important.}\\
The present document's aim is to build confidence in the software by designing 
tests that verify its correctness (compliance with the requirements). 
Validation of the inputs and outputs or the 
code usability will be among the main tests this plan will be focusing on.

\subsection{Relevant Documentation}

%\wss{Reference relevant documentation.  This will definitely include your SRS
%  and your other project documents (MG, MIS, etc)}

Through this document we will be refering to the requirement specifications for 
EOMEE[\citet{SRS2020}], in particular to its functional and nonfunctional 
requirements.

\section{Plan}
	
\subsection{Verification and Validation Team}

%\wss{You, your classmates and the course instructor.  Maybe your supervisor.}\\
The review of EOMEE will be conducted by the following parties:
\begin{itemize}
	\item The Ayers Lab members, in particular Michael Rischer, Dr.\ Paul Ayers 
	and Gabriela S\'anchez D\'iaz.
	\item Dr.\ Spencer Smith.
	\item Members of 2020 CAS741 course; in particular the following 
	responsibilities have been defined:
	\subitem Mohamed AbuElAla (Primary Reviewer)
	\subitem Seyed Parsa Tayefeh Morsal (SRS Reviewer)
	\subitem Ting-Yu Wu (VnV Reviewer)
	\subitem Xuanming Yan (MG and MIS Reviewer)
\end{itemize}

\subsection{SRS Verification Plan}

%\wss{List any approaches you intend to use for SRS verification.  This may just
%  be ad hoc feedback from reviewers, like your classmates, or you may have
%  something more rigorous/systematic in mind..}
The software requirement specifications (SRS) verification will be carried by 
the reviewers following the indications describes in the SRS verification test. 
This test is intended as a guide in the revision process.

\subsection{Design Verification Plan}

%\wss{Plans for design verification}
The modules design will be discussed with the review team members Dr.\ Paul 
Ayers and Michael Richer. It will also be presented to/evaluated by Dr.\ Smith 
and the members of 2020 CAS741 course.

\subsection{Implementation Verification Plan}

%\wss{You should at least point to the tests listed in this document and the 
%unit
%  testing plan.}
%
%\wss{In this section you would also give any details of any plans for static 
%verification of
%  the implementation.  Potential techniques include code walkthroughs, code
%  inspection, static analyzers, etc.}\\
The implementation will be verified through system and unit testing. The 
following test areas will be defined:
\begin{itemize}
	\item \textbf{Subsection} \ref{section:inputs} Input Verification : 
	includes 
	checks for the input variables types and values.
	\item \textbf{Subsection} 
	\ref{section:calculations} Energies and TDMs Evaluation: tests in 
	which the properties represented by the instance models (IM1-IM6) are 
	evaluated and compared against literature or pseudo-oracle results.
	\item \textbf{Subsection} \ref{section:srsreview} SRS Review: 
	introduces a questionnaire to guide the SRS review team.
	\item \textbf{Subsection} \ref{section:usability} Usability Verification 
	Plan: 
	tests with the purpose of identifying potential deficiencies in the 
	user-software interaction.
	\item \textbf{Subsection} \ref{section:performance} Performance: an 
	assesment 
	atemp at the numerical stability limits of the implemented IM1-IM5.
\end{itemize}

\an{The corresponding sections for the unit tests will be added once that 
section of 
the VnV plan is completed}

\subsection{Automated Testing and Verification Tools}

%\wss{What tools are you using for automated testing.  Likely a unit testing
%  framework and maybe a profiling tool, like ValGrind.  Other possible tools
%  include a static analyzer, make, continuous integration tools, test coverage
%  tools, etc.  Explain your plans for summarizing code coverage metrics.
%  Linters are another important class of tools.  For the programming language
%  you select, you should look at the available linters.  There may also be 
%tools
%  that verify that coding standards have been respected, like flake9 for
%  Python.}
%\wss{The details of this section will likely evolve as you get closer to the
%  implementation.}\\  
We will be using \href{https://docs.pytest.org/en/stable/} 
{Pytest} testing framework for tests automation  with the plugin 
\href{https://pytest-cov.readthedocs.io/en/latest/} {pytest-cov} to 
determine the code coverage percentage. Additionally, the following tools 
will be 
used to enforce compliance with Python style 
conventions for code and documentation:
\begin{itemize}
\item \href{https://www.pylint.org/} {Pylint}, 
\href{https://flake8.pycqa.org/en/latest/} {Flake8} and 
\href{https://black.readthedocs.io/en/stable/#} {Black} as Python linters. The 
last one is also a code auto-formatter. 
\item \href{https://pypi.org/project/flake8-docstrings/} {flake8-docstrings} as 
a static analysis for Python docstring conventions.
\end{itemize}
For continuous integration we will use the \href{https://travis-ci.org/} 
{Travis-CI} tool.

\subsection{Software Validation Plan}

%\wss{If there is any external data that can be used for validation, you should
%  point to it here.  If there are no plans for validation, you should state 
%that
%  here.}
To assess the validity of EOMEE results, in particular during the system tests, 
these will be compared against the ones obtained with the package for 
electronic structure calculations Gaussian[\cite{g16}]. Another source for 
validation can be the NIST Atomic Spectra 
Database[\cite{NIST_ASD}].

\section{System Test Description}
	
\subsection{Tests for Functional Requirements}

%\wss{Subsets of the tests may be in related, so this section is divided into
%  different areas.  If there are no identifiable subsets for the tests, this
%  level of document structure can be removed.}
%
%\wss{Include a blurb here to explain why the subsections below
%  cover the requirements.  References to the SRS would be good.}
The tests described in the following subsections check functional requirements 
listed in the subsection 5.1 of the 
\href{https://github.com/gabrielasd/eomee/tree/cas741/docs/SRS} {SRS}, in 
particular requirements R1 and R2 about the input variables and R4 and R5 
related to the spectroscopic properties evaluated by EOMEE.

\subsubsection{Input Verification}
\label{section:inputs}

%\wss{It would be nice to have a blurb here to explain why the subsections below
%  cover the requirements.  References to the SRS would be good.  If a section
%  covers tests for input constraints, you should reference the data constraints
%  table in the SRS.}
		
\paragraph{Input type and value}

\begin{enumerate}

\item{input-type:IT\\}

Control: Automatic
					
Initial State: N/A
					
Input: Two sets of inputs, corresponding to the one-electron integral, 
two-electron integral, one-electron reduced density matrix and two-electron 
reduced density matrix, will be provided according to 
Table~\ref{Table:typeerror}:

\begin{table}[h!]
	\centering
	\noindent \begin{tabular}{l l l } 
		\toprule		
		\textbf{Input set} & \textbf{S1} & \textbf{S2}\\
		\midrule 
		$h_{pq}$& tuple & 2-idx numpy array\\
		$v_{pqrs}$& 4-idx numpy array& 4-idx numpy array\\
		$\gamma_{pq}$& 2-idx numpy array& 2-idx list\\
		$\Gamma_{pqrs}$& 4-idx numpy array& 4-idx numpy array\\
		\textbf{Wrong}& tuple variable& list variable\\
		\bottomrule
	\end{tabular}
	\caption{Type of the input variables}
	\label{Table:typeerror}
\end{table}

The variables in the table will be defined as:\\
tuple: ((1,2), (3,4))\\
2-idx numpy array: numpy.random.rand(4, 4).astype(float)\\
2-idx list: [[1.0, 2.0],[3.0, 4.0]]\\
4-idx numpy array: numpy.random.rand(4, 4, 4, 4).astype(float)
	
Output: A type-error will be raised in each case. 

Test Case Derivation: N/A
					
How test will be performed: The test module will feed the inputs, as specified 
in Table~\ref{Table:typeerror}, to the EOMEE code and verify that a type-error 
message is raised.

					
\item{input-value:IV\\}

Control: Automatic
					
Initial State: N/A
					
Input: Five input sets, corresponding to the one-electron integral, 
two-electron integral, one-electron reduced density matrix and two-electron 
reduced density matrix, as described in Table~\ref{Table:valueerror}.
\begin{table}[h!]
	\centering
	\noindent \begin{tabular}{l c c l l c} 
		\toprule		
		\textbf{Input set} & \textbf{S1} & \textbf{S2}& 
		\textbf{S3} & \textbf{S4}& \textbf{S5}\\
		\midrule 
		$h_{pq}$& $v_{pqrs}$ & $h_{pq}$& $A_{(n \times m)}$& 
		$A_{(m \times m)}$& $\gamma_{pq}$\\
		$v_{pqrs}$& $h_{pq}$& $v_{pqrs}$&$B_{(n \times m 
		\times m \times m)}$& $B_{(n \times n 
		\times n \times n)}$& $\Gamma_{pqrs}$\\
		$\gamma_{pq}$& $\gamma_{pq}$& $\Gamma_{pqrs}$& $A_{(n \times m)}$& 
		$C_{(n \times n)}$& $h_{pq}$ 
		\\
		$\Gamma_{pqrs}$& $\Gamma_{pqrs}$& $\gamma_{pq}$&$B_{(n \times m \times 
		m \times m)}$& $D_{(n \times n \times n \times 
		n)}$& $v_{pqrs}$\\
		\textbf{Wrong}& \multicolumn{2}{l}{array dimensions} 
		&\multicolumn{2}{c}{array shape} & RDMs constrains\\
		\bottomrule
	\end{tabular}
	\caption{Test input variables definition.}
	\label{Table:valueerror}
\end{table}

The variables in Table~\ref{Table:valueerror} will be defined as:\\
$h_{pq}$: numpy.random.rand(4, 4).astype(float)\\
$v_{pqrs}$ and $B_{(n \times n	\times n \times n)}$: numpy.random.rand(4, 4, 
4, 4).astype(float)\\
$\gamma_{pq}$ and $C_{n \times n}$: numpy.asarray([[1,0,0,0], [0,0,0,0], 
[0,0,1,0], [0,0,0,0]])\\
$\Gamma_{pqrs}$ and $D_{(n \times n	\times n \times n)}$: $\gamma_{pr} \otimes 
\gamma_{qs} -  \gamma_{ps} \otimes 
\gamma_{qr}$\\
$A_{(n \times m)}$: numpy.random.rand(4, 3).astype(float)\\
$B_{(n \times m	\times m \times m)}$: numpy.random.rand(4, 3, 3, 
3).astype(float)\\
$A_{(m \times m)}$: numpy.random.rand(6, 6).astype(float)

Output: The EOMEE code must raise a value error for each case specified in 
Table~\ref{Table:valueerror}.

Test Case Derivation: Input cases S1 and S2 target incorrect dimensions in the 
electron integrals or RDMs. S3 verifies that $h_{pq}$ and $\gamma_pq$ are 
square matrices, while $v_{pqrs}$ and $\Gamma_{pqrs}$ must be tensors, each 
with equal numbers of rows and columns. Mismatching number of spinorbitals in 
input parameters (given by the array shapes) is verified by the test case S4. 
The last column input set targets the requirement R2 in the SRS relative to the 
physical constraints of the RDMs.

How test will be performed: The test cases in Table~\ref{Table:valueerror} are 
passed to the EOMEE code by the testing framework which will also verify that a 
value-error message is raised.

\end{enumerate}

\subsubsection{Energies and TDMs Evaluation}
\label{section:calculations}

\paragraph{Transition energies\\}

The tests described under this section require the one- and 
two-electron integrlas for the N-electron, M-atom system to be stored in the 
directory \textit{test/data} as \href{https://numpy.org/} {NumPy}'s binary 
files (.npy). Unless otherwise 
stated, for each implemented EOM method (IM1-IM5, 
\href{https://github.com/gabrielasd/eomee/tree/cas741/docs/SRS} {SRS} 
subsection 4.2.5), the lowest (positive valued) 
transition energy will be taken and compared against an "expected" value 
(generally from a Gaussian[\cite{g16}] calculation). 
Because at this point we are not targeting the performance of the 
methods, only small systems with at most 5 electrons and 10 molecular orbitals 
(MOs) 
were included in the test cases bellow. We will use the absolute error 
($|E_{IM\#} - E_{expected}|$) to measure the method's accuracy.
%Electronic structure calculations are said to have chemical accuracy if their 
%prediction error is around 1 kcal/mol (0.043 eV) relative to the experiment. 
%Reliable methods for excitation energies computation such as 
%EOM-CCSD~\cite{Stanton1993} provide an accuracy of 
%0.1–0.2 eV (absolute error) for sinlge-excitations.\\


\begin{enumerate}
	
	\item{correct-lowest-transition:LT\\}
	
	Control: Automatic
	
	Initial State: N/A
	
	Input: The one- and two-electron integrals corresponding to the systems 
	presented in Table~\ref{table:energies}, the tolerance value $1.0 \times 
	10^{-6}$.\\
	
	Table~\ref{table:energies} contains the test cases 
	description. All systems have closed-shell electronic configurations except 
	the boron atom (B).

	\begin{table}[h!]
		\centering
		\begin{tabular}{l c c c c}
			\toprule
			\textbf{File} & IM &\textbf{Nbasis} & \textbf{Nocc}& 
			\textbf{Expected}\\
			\midrule
			B (STO-3G) & 1 & 5& (3, 2)& $-\epsilon_{HOMO}$\\
			B (STO-3G) & 2 & 5& (3, 2)& $\epsilon_{LUMO}$\\
			He (cc-pVDZ)& 1& 5& (1, 1)& $-\epsilon_{HOMO}$\\
			He (cc-pVDZ)& 2& 5& (1, 1)& $\epsilon_{LUMO}$\\
			$HeH^{+}$ (STO-3G)& 1 & 2& (1, 1)& $-\epsilon_{HOMO}$\\
			$HeH^{+}$ (STO-3G)& 2 & 2& (1, 1)& $\epsilon_{LUMO}$\\
			$HeH^{+}$ (STO-3G)& 3 & 2& (1, 1)& 0.91123209\\
			$H_2$ (STO-6G)& 4 & 2& (1, 1)& 1.83843430\\			
		\end{tabular}
		\caption{Input variables. The reference energy values (last column) are 
		in Hartree.}
		\label{table:energies}
	\end{table}

	\begin{table}[h!]
		\centering
		\begin{tabular}{l c c}
			\toprule
			\textbf{File} & $\epsilon_{HOMO}$ & $\epsilon_{LUMO}$\\
			\midrule
			B (STO-3G) &-0.20051823& 0.29136562\\
			He (cc-pVDZ)& -0.91414765& 1.39744193\\
			$HeH^{+}$ (STO-3G)& -1.52378328 & -0.26764028\\		
		\end{tabular}
		\caption{Frontier molecular orbital energies in Hartree.}
		\label{table:koopman}
	\end{table}
	
	Output: Eigenvalues (and eigenvectors) solution to the IMs.
	
	Test Case Derivation: 
	The one- and two-reduced density matrices required  as input by the EOMEE 
	methods will be generated from the number of electrons (Nocc) and MOs 
	(Nbasis). Because the ground state wavefunction ($\Psi_0$) for the 
	systems in Table~\ref{table:energies} will be modeled 	by a single Slater 
	determinant (specifically the lowest energy Hartree-Fock Slater 
	determinant), the one- and two-RDMs can be defined as:
	\begin{align*}
	\gamma_{pq} &=\Bigg\{
	\begin{matrix}
	\delta_{pq}& p,q \in \text{occupied MO}\\
	0 & \text{virtual MO}
	\end{matrix}\\
	\Gamma_{pqrs} &= \frac{1}{\sqrt{2}}(\gamma_{pr}\gamma_{qs} - 
	\gamma_{ps}\gamma_{qr})
	\end{align*} 
	
	How test will be performed:
	The testing framework will pass the electron integrals and density matrices 
	to the EOM method (IM column in Table~\ref{table:energies}). From the 
	solutions, the lowest transition energy will be determined and compared 
	against the "expected" value. If the absolute error is less or 
	equal than the tolerance criteria, the test will pass.

	
\end{enumerate}


\paragraph{Transition Density Matrices}

\begin{enumerate}
	
	\item{matrix-symmetry:TS\\}
	
	Control: Automatic
	
	Initial State: The EOM method (IM1-IM5) has been solved.
	
	Input: The following input arrays located in the \textit{test/data} folder:
	\begin{itemize}
		\item be\_sto3g\_twoint\_genzd\_anti.npy
		\item 2dm\_be\_sto3g\_genzd\_anti.npy
		\item be\_sto3g\_oneint\_genzd.npy
		\item 1dm\_be\_sto3g\_genzd.npy
	\end{itemize}
	The expansion coefficients matrix and the lowest nonzero excited state 
	index. 
	
	Output: The TDM for the selected state.
	
	Test Case Derivation: The equations to evaluate the TDMs corresponding to 
	each EOM method can be found in the 
	\href{https://github.com/gabrielasd/eomee/tree/cas741/docs/SRS} 
	{SRS} (subsection 4.2.5, IM6).
	
	How test will be performed: 
	The test framework will pass the specified inputs to a function that 
	evaluates the EOMEE methods and TDMs. The selected TDM and its transpose 
	will be compared using NumPy's allclose method. The test will pass if the 
	matrix is asymmetric.
	
%	\item{test-id2\\}
%	
%	Control: Manual versus Automatic
%	
%	Initial State: 
%	
%	Input: 
%	
%	Output: \wss{The expected result for the given inputs}
%	
%	Test Case Derivation: \wss{Justify the expected value given in the Output 
%		field}
%	
%	How test will be performed: 
	
\end{enumerate}


\subsection{Tests for Nonfunctional Requirements}

%\wss{The nonfunctional requirements for accuracy will likely just reference the
%  appropriate functional tests from above.  The test cases should mention
%  reporting the relative error for these tests.}
%
%\wss{Tests related to usability could include conducting a usability test and
%  survey.}
The tests described in the following subsections verify the nonfunctional 
requirements listed in the subsection 5.2 of the 
\href{https://github.com/gabrielasd/eomee/tree/cas741/docs/SRS} {SRS}.

\subsubsection{SRS Review}
\label{section:srsreview}
		
%\paragraph{Title for Test}

\begin{enumerate}

\item{SRS\_questionnaire:SQ}

Type: Manual
					
Initial State: N/A
					
Input/Condition: 
The the SRS document[\cite{SRS2020}] will be provided to Dr.\ Ayers, Dr.\ Smith, 
Mohamed AbuElAla and Gabriela S\'anchez along with a questionnaire (to be found 
in the \hyperref[section:appendix]{Appendix} section).
					
Output/Result: GitHub issues 
					
How test will be performed: 
The reviewers will read the document and questions and  report any problems or 
inconsistencies.

\end{enumerate}

\subsubsection{Usability verification}
\label{section:usability}

\paragraph{Portability and usability tests}

\begin{enumerate}
	
	\item{installability-test:PT}
	
	Type: Manual
	
	Initial State: The may or may not be some Python program and modules 
	already installed.
	
	Input/Condition: Internet connection and some web browser available.
	
	Output/Result: EOMEE module installed or issue reported in GitHub.
	
	How test will be performed: 
	The GitHub repository will be cloned by Gabriela S\'anchez in different 
	platforms (Ubuntu Focal 
	Fossa, Mac OS Catalina and Windows 10). The installation will follow the 
	instructions in the README.md file located on the repository root folder. 
	Upon installation completion the system and unit tests will be ran. 

	\item{usability-test:UT\\}
	\label{sec:usability}
	
	Type: Manual.
	
	Initial State: The EOMEE code may or not be installed on the user's 
	computer.
	
	Input/Condition: Members of the test team, the link to the project's GitHub 
	repository and the following example input files for the beryllium atom:
	\begin{itemize}
		\item be\_sto3g\_twoint\_genzd\_anti.npy
		\item 2dm\_be\_sto3g\_genzd\_anti.npy
		\item be\_sto3g\_oneint\_genzd.npy
		\item 1dm\_be\_sto3g\_genzd.npy
	\end{itemize}
	These files will be located in the \textit{test/data} folder.
	
	Output/Result: Users response to the usability survey found in the Appendix 
	section.
	
	How test will be performed: The users will clone the repository and install 
	the program. Using the provided input files they will choose to evaluate 
	one of the following properties for Be: ionization potential, electron 
	affinity or electron excitation. Additionally, they will determine the TDMs 
	for their selected method. After completing these tasks the test team will 
	fill out the survey.
	
	
\end{enumerate}


\subsubsection{Performance}
\label{section:performance}

\paragraph{Numerical instability}

\begin{enumerate}
	
	\item{illconditioning-test:NIT}
	
	Type: automatic
	
	Initial State: N/A
	
	Input/Condition: A matrix orthogonalization methods (strings 
	\textit{symmetric} or \textit{asymmetric}), threshold values in the range 
	$1.0 \times 10^{-4} - 1.0 \times 10^{-9}$, the example input files from the
	\hyperref[section:usability]{usability-test} and the instance models IM1 to 
	IM5 (\href{https://github.com/gabrielasd/eomee/tree/cas741/docs/SRS} {SRS} 
	subsection 4.2.5).
	
	Output/Result: A graph showing how the lowest nonzero value for instance 
	models IM1-IM5 changes with the tolerance value.
	
	How test will be performed: The test framework will pass the inputs to 
	EOMEE and a graph will be generated.
	
\end{enumerate}


\subsection{Traceability Between Test Cases and Requirements}

%\wss{Provide a table that shows which test cases are supporting which
%  requirements.}
Table~\ref{table:traceab} shows what requirement specificationns the functional and nonnfunnctional system tests address:

\begin{table}[ht]
	\centering
	\begin{tabular}{|l|c|c|c|c|c|c|c|c|}
		\hline
		& IT& IV& LT& TS& SQ& PT& UT& NIT\\	\hline	
	  R1& X& X& & & & & &\\ \hline
	  R2&  & X& & & & & &\\ \hline
	  R3&  & & & & & & &\\ \hline
	  R4&  &  & X& X& & & &\\ \hline
	  R5&  &  & X& X& & & &\\ \hline
	  R6&  &  & & & & & &\\ \hline
	  Reusable& & & & & & & & \\ \hline
	  Usable& & & & &  & X& X& \\ \hline
	  Portable& & & & &  & X& & \\ \hline
	  Correct& & & X& X&  X&  & & \\ \hline
	\end{tabular}
	\caption{Traceability for system tests and requirements}
	\label{table:traceab}
\end{table}

\section{Unit Test Description}

\wss{Reference your MIS and explain your overall philosophy for test case
  selection.}  
\wss{This section should not be filled in until after the MIS has
  been completed.}

\subsection{Unit Testing Scope}

\wss{What modules are outside of the scope.  If there are modules that are
  developed by someone else, then you would say here if you aren't planning on
  verifying them.  There may also be modules that are part of your software, but
  have a lower priority for verification than others.  If this is the case,
  explain your rationale for the ranking of module importance.}

\subsection{Tests for Functional Requirements}

\wss{Most of the verification will be through automated unit testing.  If
  appropriate specific modules can be verified by a non-testing based
  technique.  That can also be documented in this section.}

\subsubsection{Module 1}

\wss{Include a blurb here to explain why the subsections below cover the module.
  References to the MIS would be good.  You will want tests from a black box
  perspective and from a white box perspective.  Explain to the reader how the
  tests were selected.}

\begin{enumerate}

\item{test-id1\\}

Type: \wss{Functional, Dynamic, Manual, Automatic, Static etc. Most will
  be automatic}
					
Initial State: 
					
Input: 
					
Output: \wss{The expected result for the given inputs}

Test Case Derivation: \wss{Justify the expected value given in the Output field}

How test will be performed: 
					
\item{test-id2\\}

Type: \wss{Functional, Dynamic, Manual, Automatic, Static etc. Most will
  be automatic}
					
Initial State: 
					
Input: 
					
Output: \wss{The expected result for the given inputs}

Test Case Derivation: \wss{Justify the expected value given in the Output field}

How test will be performed: 

\item{...\\}
    
\end{enumerate}

\subsubsection{Module 2}

...

\subsection{Tests for Nonfunctional Requirements}

\wss{If there is a module that needs to be independently assessed for
  performance, those test cases can go here.  In some projects, planning for
  nonfunctional tests of units will not be that relevant.}

\wss{These tests may involve collecting performance data from previously
  mentioned functional tests.}

\subsubsection{Module ?}
		
\begin{enumerate}

\item{test-id1\\}

Type: \wss{Functional, Dynamic, Manual, Automatic, Static etc. Most will
  be automatic}
					
Initial State: 
					
Input/Condition: 
					
Output/Result: 
					
How test will be performed: 
					
\item{test-id2\\}

Type: Functional, Dynamic, Manual, Static etc.
					
Initial State: 
					
Input: 
					
Output: 
					
How test will be performed: 

\end{enumerate}

\subsubsection{Module ?}

...

\subsection{Traceability Between Test Cases and Modules}

\wss{Provide evidence that all of the modules have been considered.}
				
\bibliographystyle{plainnat}

\bibliography{../../refs/References}

\newpage

\section{Appendix}
\label{section:appendix}

%This is where you can place additional information.

\subsection{Symbolic Parameters}

The definition of the test cases will call for SYMBOLIC\_CONSTANTS.
Their values are defined in this section for easy maintenance.

\subsection{Usability Survey Questions}

\begin{enumerate}
	\item Were the installation instructions for your platform clear? Could you 
	successfully install EOMEE by following them?
	\item Was the code's documentation helpful when learning how to run the 
	calculations?
	\item Did you look at the tests files when learning how to run the 
	calculations? Were they helpful?
	\item Could you successfully run an excited state calculation?
	\item Could you successfully determine the transition density matrices for 
	your chosen EOM method?
	\item Was the naming of the functions descriptive (it was clear the 
	task they were supposed to perform)?
	\item Do you have any improvement suggestions regarding the documentation 
	or code?
	\item How would you rate the learning curve for this software?(Very easy, 
	easy, hard, very hard)
\end{enumerate}

\subsection{SRS Questionnaire}
Because the members of the VnV team might have limited time to review the 
document some of the question bellow have been specifically assigned:
\begin{enumerate}
	\item Are the energy units in the Table of units (section 1.2) appropriate 
	for the physical problems EOMEE addresses?
	\item Was the Table of Symbols (section 1.2) complete? 
	\item Does the used notation in the Table of Symbols (section 1.2) matches 
	the one generally used in the literature? [Dr.\ Ayers]
	\item Were the abbreviations and acronyms (section 1.3) properly used 
	through the document?
	\item Is Figure 1 for the system context (section 3.1) along with its 
	description correct? Do the listed user and software responsibilities make 
	sense? [Dr. Smith and Mohamed AbuElAla]
	\item Was the description of the problem EOMEE intends to solve clear? Were 
	all the terminology definitions needed for the understanding of subsequent 
	sections included? (sections 4.1 and 4.1.1) [Dr.\ Smith 
	and Mohamed AbuElAla]
	\item Were the terminology definitions listed in section 4.1.1 correct? 
	[Dr.\ Ayers]
	\item Is section 4.1.2 (Physical System Description) clear? Do you have any 
	suggestions that might help improve the understanding of EOMEE's physical 
	system? [Mohamed AbuElAla]
	\item Were all assumptions pertinent to the EOM models included? Are the 
	ones listed correct?(section 4.2.1) [Dr.\ Ayers]
	\item Were the connections between the assumptions and the appropriate IM, 
	GS, DD, T or LC included? (section 4.2.1)
	\item Were the defined theoretical models the correct general equations 
	that EOMEE is based on? Were the descriptions and derivations appropriate? 
	(section 4.2.2) [Dr.\ Ayers]
	\item In section 4.2.3, are the definitions of the transition operator and 
	EOM approximations correct? Are the descriptions and derivations 
	appropriate? [Dr.\ Ayers]
	\item Are DD2, DD3, DD4 and DD5 properly described? Are the presented 
	symmetry properties of DD3 and DD5 correct? (section 4.2.4) [Dr.\ Ayers]
	\item Are the IMs inputs and outputs appropriate? (section 4.2.5) [Dr.\ 
	Smith]
	\item After reading the Input Data Constraints section (4.2.6), are the 
	physical constraints presented in the Table 1 correct? Are the typical 
	values assigned to the one- and two-electron integrals reasonable? What 
	changes do you suggested making to the values presented in this table? 
	[Dr.\ Ayers]
	\item Do the output constraints presented in Table 3 (section 4.2.7) make 
	sense? If not, could you suggest changes to be made? [Dr.\ Ayers]
	\item Do the listed Functional and Nonfunctional Requirements make sense 
	(sections 5.1 and 5.2 respectively) [Dr.\ Smith]
	\item Were the cross-references made correctly through the document? 
	[Mohamed AbuElAla]
	
\end{enumerate}
These questions are inspired on the SRS review plans by Malavika Srinivasan and 
Spencer Smith[\cite{malavika}]. 
\end{document}