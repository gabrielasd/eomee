\documentclass[12pt, titlepage]{article}

\usepackage{amsmath, mathtools}

\usepackage[round]{natbib}
\usepackage{amsfonts}
\usepackage{amssymb}
\usepackage{graphicx}
\usepackage{colortbl}
\usepackage{xr}
\usepackage{hyperref}
\usepackage{longtable}
\usepackage{xfrac}
\usepackage{tabularx}
\usepackage{float}
\usepackage{siunitx}
\usepackage{booktabs}
\usepackage{multirow}
\usepackage[section]{placeins}
\usepackage{caption}
\usepackage{fullpage}

\hypersetup{
bookmarks=true,     % show bookmarks bar?
colorlinks=true,       % false: boxed links; true: colored links
linkcolor=red,          % color of internal links (change box color with linkbordercolor)
citecolor=blue,      % color of links to bibliography
filecolor=magenta,  % color of file links
urlcolor=cyan          % color of external links
}

\usepackage{array}

%\externaldocument{../../SRS/SRS}
%\externaldocument{../MG/MG}

%% Comments

\usepackage{color}

\newif\ifcomments\commentstrue

\ifcomments
\newcommand{\authornote}[3]{\textcolor{#1}{[#3 ---#2]}}
\newcommand{\todo}[1]{\textcolor{red}{[TODO: #1]}}
\else
\newcommand{\authornote}[3]{}
\newcommand{\todo}[1]{}
\fi

\newcommand{\wss}[1]{\authornote{blue}{SS}{#1}} 
\newcommand{\plt}[1]{\authornote{magenta}{TPLT}{#1}} %For explanation of the template
\newcommand{\an}[1]{\authornote{cyan}{Author}{#1}}

%% Common Parts

\newcommand{\progname}{ProgName} % PUT YOUR PROGRAM NAME HERE %Every program
                                % should have a name


\begin{document}

\title{Module Interface Specification for EOMEE}

\author{Gabriela S\'anchez D\'iaz}

\date{\today}

\maketitle

\pagenumbering{roman}

\section{Revision History}

\begin{tabularx}{\textwidth}{p{3cm}p{2cm}X}
\toprule {\bf Date} & {\bf Version} & {\bf Notes}\\
\midrule
20-11-2020 & 1.0 & MIS first draft\\
%Date 2 & 1.1 & Notes\\
\bottomrule
\end{tabularx}

~\newpage

\section{Symbols, Abbreviations and Acronyms}

See SRS Documentation at 
\href{https://github.com/gabrielasd/eomee/tree/cas741/docs/SRS/}{SRS}.\\
%\wss{Also add any additional symbols, abbreviations or acronyms}\\
Additionaly, the following abbreaviations were used:
\noindent \begin{longtable*}{l p{5cm}} \toprule
	\textbf{abbreviation} & \textbf{description}\\
	\midrule 
	\textit{N} & Number of electrons\\ 
	orthog & Matrix orthogonalization method\\
	tol & Tolerance\\
	nspino & Number of spin orbital basis\\
	lhs & Left-hand-side\\
	rhs & Right-hand-side\\
	neigs & Number of eigenvalues\\
\end{longtable*}


\newpage

\tableofcontents

\newpage

\pagenumbering{arabic}

\section{Introduction}

The following document details the Module Interface Specifications of EOMEE, a 
set of tools to implement and solve the Equation-of-Motion methods for excited 
states.
%\wss{Fill in your project name and description}

Complementary documents include the System Requirement Specifications
and Module Guide.  The full documentation and implementation can be
found at \url{https://github.com/gabrielasd/eomee/tree/cas741}.  

\section{Notation}

%\wss{You should describe your notation.  You can use what is below as
%  a starting point.}

The structure of the MIS for modules comes from \citet{HoffmanAndStrooper1995},
with the addition that template modules have been adapted from
\cite{GhezziEtAl2003}.  The mathematical notation comes from Chapter 3 of
\citet{HoffmanAndStrooper1995}.  For instance, the symbol := is used for a
multiple assignment statement and conditional rules follow the form $(c_1
\Rightarrow r_1 | c_2 \Rightarrow r_2 | ... | c_n \Rightarrow r_n )$.

The following table summarizes the primitive data types used by EOMEE. 

\begin{center}
\renewcommand{\arraystretch}{1.2}
\noindent 
\begin{tabular}{l l p{7.5cm}} 
\toprule 
\textbf{Data Type} & \textbf{Notation} & \textbf{Description}\\ 
\midrule
character & char & a single symbol or digit\\
integer & $\mathbb{Z}$ & a number without a fractional component in (-$\infty$, $\infty$) \\
natural number & $\mathbb{N}$ & a number without a fractional component in [1, $\infty$) \\
real & $\mathbb{R}$ & any number in (-$\infty$, $\infty$)\\
\bottomrule
\end{tabular} 
\end{center}

\noindent
The specification of EOMEE \ uses some derived data types: sequences, strings, 
and tuples. Sequences are lists filled with elements of the same data type. 
Strings are sequences of characters. Tuples contain a list of values, 
potentially of different types. In addition, EOMEE \ uses functions, which
are defined by the data types of their inputs and outputs. Local functions are
described by giving their type signature followed by their specification.\\
The following simplifications will be made in the mathematical notation for the 
sake of understandability:
\begin{itemize}
	\item seq($l_1$,$l_2$,...,$l_n$:T), will be used instead of 
	sequence 
	[$l_1$,$l_2$,...,$l_n$] of type T. For example seq($n,m:\mathbb{R}$), where 
	$n,m > 0$, would map to sequence [$n$, $m$] of type $\mathbb{R}$.This 
	type will generally be used to indicate NumPy.ndarray data types.
	\item Variables that are of type sequence will be denoted in bold font, 
	i.e, the parameter $\textbf{x}$ denotes a sequence. 
	\item Subscripts will be used for indexing sequences, for instance, $x_i$ 
	will represent the $i$th element of $\textbf{x}$, the same as $x[i]$ from 
	\citet{HoffmanAndStrooper1995}.
	\item str will be used instead of string.
	\item bool will be used instead of boolean.
\end{itemize}
Also, the absence of value will be defined by Python's data type NoneType, 
denoted as None.


\section{Module Decomposition}

The following table is taken directly from the Module Guide document for this project.

\begin{table}[h!]
	\centering
	\begin{tabular}{p{0.3\textwidth} p{0.3\textwidth} p{0.3\textwidth}}
		\toprule
		\textbf{Level 1} & \textbf{Level 2} & \textbf{Level 3}\\
		\midrule
		
		{Hardware-Hiding Module} & ~ & ~\\
		\midrule
		
		\multirow{7}{0.3\textwidth}{Behaviour-Hiding Module} & Control 
		& ~\\
		& Input & \\
		& Integrals& \\
		& RDMs& \\
		& EOM Interface& IP EOM; EA EOM; DIP EOM; DEA EOM; Excitation EOM\\
		& Output & \\
		%& ?\\
		\midrule
		
		\multirow{1}{0.3\textwidth}{Software Decision Module} & {Solver}&\\
		%& Solver &\\
		%& ?\\
		\bottomrule
		
	\end{tabular}
	\caption{Module Hierarchy}
	\label{TblMH}
\end{table}

%\section{MIS of \wss{Module Name}} \label{Module} \wss{Use labels for
%  cross-referencing}
\section{MIS of Control Module} \label{mDriver}

%\wss{You can reference SRS labels, such as R\ref{R_Inputs}.}
%
%\wss{It is also possible to use \LaTeX for hypperlinks to external documents.}

\subsection{Module}

%\wss{Short name for the module}
main

\subsection{Uses}
input (\ref{mInput}), Integrals (\ref{mIntegrals}), WfnRDMs (\ref{mRDMs}), 
EOMIP (\ref{mEOMIP}), EOMEA (\ref{mEOMEA}), EOMExc (\ref{mEOMExc}), EOMDIP 
(\ref{mEOMDIP}), EOMDEA (\ref{mEOMDEA}),  solver (\ref{mGEP}), output 
(\ref{mOutput})


\subsection{Syntax}

\subsubsection{Exported Constants}
None

\subsubsection{Exported Access Programs}

\begin{center}
\begin{tabular}{p{2cm} p{4cm} p{4cm} p{2cm}}
\hline
\textbf{Name} & \textbf{In} & \textbf{Out} & \textbf{Exceptions} \\
\hline
%main & $\mathbb{Z} \lor (n1, 
%n2:\mathbb{Z})$, str, str, str, str, bool& - & - \\
main & str& - & - \\
\hline
\end{tabular}
\end{center}

\subsection{Semantics}

\subsubsection{State Variables}

%\wss{Not all modules will have state variables.  State variables give the 
%module
%  a memory.}
None

\subsubsection{Environment Variables}

%\wss{This section is not necessary for all modules.  Its purpose is to capture
%  when the module has external interaction with the environment, such as for a
%  device driver, screen interface, keyboard, file, etc.}
%infiles: 5 strings corresponding to the number of particles (npart), a 
%selection of the EOM method and the files containing the one- and two-electron 
%integrlas ($h_{pq}$ and $v_{pqrs}$) and one- and two-RDMs ($\gamma_{pq}$ and 
%$\Gamma_{pqrs}$).
None

\subsubsection{Assumptions}

%\wss{Try to minimize assumptions and anticipate programmer errors via
%  exceptions, but for practical purposes assumptions are sometimes 
%appropriate.}
%The input files contain $h_{pq}$, $v_{pqrs}$, $\gamma_{pq}$ and 
%$\Gamma_{pqrs}$ 
%stored as sequences of sequences (NumPy arrays) and are loaded in that 
%specific 
%order.\\
None

\subsubsection{Access Routine Semantics}

\noindent main():
\begin{itemize}
\item transition: The following steps are performed:\\
%Get from the user the input parameters \textit{npart} ($\mathbb{Z} \lor (n1, 
%n2:\mathbb{Z})$), \textit{$h_{pq}$} (2D NumPy.ndarray), \textit{$v_{pqrs}$} 
%(4D 
%NumPy.ndarray), 
%\textit{$\gamma_{pq}$} (2D NumPy.ndarray), \textit{$\Gamma_{pqrs}$} (4D 
%NumPy.ndarray), \textit{eom} 
%(str $\in \{``ip", ``dip", ``ea", ``dea", ``exc"\}$) and \textit{tdm} (bool 
%$\in$ 
%\{True, 
%False\}). The last two parameters selects the EOM method and whether the TDMs 
%are evaluated or not.\\
Get a file containing the input parameters from the user (inputFile).\\
\newline Parse the file's content and verify all required input parameters are 
present.\\
%*parse\_inputfile(inputFile)\\
%*check\_inputs()
\newline Load and verify the electron integrals (\textbf{h},\textbf{v}) and 
RDMs ($\boldsymbol{\gamma}$, $\bold{\Gamma}$)\\
%>Integrals($h_{pq}, v_{pqrs}$)
\newline Define an EOM type equation from the parameters \textbf{h},\textbf{v}, 
$\boldsymbol{\gamma}$ and $\bold{\Gamma}$.\\
\newline Solve the EOM eigenvalue problem and evaluate the TDMs\\
%$\Delta E$, \textit{coeffs} := solve.dense(EOM.lhs, EOM.rhs, ortog, tol)\\
%Evaluate the TDMs:\\
%tdms := EOM.tdm(coeffs)\\
\newline Output the results of the computations:\\
%output(filename1, $\Delta E$, \textit{coeffs})\\
%output(filename2, tdms)
%\item output: \wss{if appropriate} 
\item exception: None
\end{itemize}

%\wss{A module without environment variables or state variables is unlikely to
%  have a state transition.  In this case a state transition can only occur if
%  the module is changing the state of another module.}
%
%\wss{Modules rarely have both a transition and an output.  In most cases you
%  will have one or the other.}

\subsubsection{Local Functions}

%\wss{As appropriate} \wss{These functions are for the purpose of specification.
%  They are not necessarily something that is going to be implemented
%  explicitly.  Even if they are implemented, they are not exported; they only
%  have local scope.}
None
\newpage

\section{MIS of Input Module} \label{mInput}

\subsection{Module}

input

\subsection{Uses}
None

\subsection{Syntax}

\subsubsection{Exported Constants}

\subsubsection{Exported Access Programs}

\begin{center}
	\begin{tabular}{p{2.7cm} p{4cm} p{4cm} p{2cm}}
		\hline
		\textbf{Name} & \textbf{In} & \textbf{Out} & \textbf{Exceptions} \\
		\hline
		parse\_inputfile& str& ParsedParams & FileNotFoundError \\
		check\_inputs& ParsedParams& - & FileNotFoundError, ValueError \\
		\textit{N}& & $(n1, n2:\mathbb{Z})$ & \\
%		\textit{nso}& & $\mathbb{Z}$ & \\
		\textit{one\_int\_file}& & str & \\
		\textit{two\_int\_file}& & str & \\
		\textit{dm1\_file}& & str & \\
		\textit{dm2\_file}& & str & \\
		\textit{eom}& & str & \\
		\textit{orthog}& & str & \\
		\textit{tol}& & $\mathbb{R}$ & \\
%		load\_integrals& str, str& seq($m,m:\mathbb{R}$), 
%		seq($m,m,m,m:\mathbb{R}$) & -\\
%		load\_rdms& str, str& seq($m,m:\mathbb{R}$), seq($m,m,m,m:\mathbb{R}$) 
%		& -\\
%		FileNotFoundError\\
%		verify\_integrals& one\_int, two\_int& - & TypeError, ValueError \\
%		load\_wfnparams& $\mathbb{Z} \lor (n1, n2:\mathbb{Z})$, $\mathbb{Z}$, 
%		str, str& WfnT & FileNotFoundError\\		
% 		verify\_rdms& dm1, dm2, npart& -& TypeError, ValueError\\
		\hline
	\end{tabular}
\end{center}

\subsection{Semantics}

\subsubsection{State Variables}

%\wss{Not all modules will have state variables.  State variables give the 
%module
%	a memory.}
\textit{N}: $\mathbb{Z} \lor (n1, n2:\mathbb{Z})$\\
%\textit{nso}: $\mathbb{Z}$\\
\textit{one\_int\_file}: str\\
\textit{two\_int\_file}: str\\
\textit{dm1\_file}: str\\
\textit{dm2\_file}: str\\
\textit{eom}: str $\in \{"ip", "dip", "ea", "dea", "exc"\}$ which selects the 
EOM method\\
\textit{orthog}: str $\in \{``symmetric", ``asymmetric"\}$\\
\textit{tol}: $\mathbb{R} > 0$\\

\subsubsection{Environment Variables}

%\wss{This section is not necessary for all modules.  Its purpose is to capture
%	when the module has external interaction with the environment, such as for a
%	device driver, screen interface, keyboard, file, etc.}
inputFile: string representing a file or file path.\\
%one\_int\_file: binary file in NumPy .npy format, contains an array.\\
%two\_int\_file: binary file in NumPy .npy format, contains an array.

\subsubsection{Assumptions}

%\wss{Try to minimize assumptions and anticipate programmer errors via
%	exceptions, but for practical purposes assumptions are sometimes 
%	appropriate.}
The first function called will be parse\_infile, followed by check\_inputs.
\subsubsection{Access Routine Semantics}
%\wss{A module without environment variables or state variables is unlikely to
%	have a state transition.  In this case a state transition can only occur if
%	the module is changing the state of another module.}
%
%\wss{Modules rarely have both a transition and an output.  In most cases you
%	will have one or the other.}

\noindent parse\_infile(\textit{filename}):
\begin{itemize}
	\item transition: The input file \textit{filename} is read sequentially 
	and the state variables get assigned
	\item output: \textit{out} := ParsedParams
	\item exception: FileNotFoundError 
\end{itemize}

\noindent check\_inputs(\textit{ParsedParams}):
\begin{itemize}
	%	\item transition: \wss{if appropriate} 
	\item output: None 
	\item exception: \textit{exc} :=  
\end{itemize}
\noindent \begin{longtable*}[l]{l l}
	$\neg$(\textit{N} $\in (n1, n2:\mathbb{Z})$) & $\Rightarrow$ TypeError\\
	``one\_int\_file" not in working directory & $\Rightarrow$ 
	FileNotFoundError\\
	``two\_int\_file" not in working directory & $\Rightarrow$ 
	FileNotFoundError\\
	``dm1\_file" not in working directory & $\Rightarrow$ 
	FileNotFoundError\\
	``dm2\_file" not in working directory & $\Rightarrow$ 
	FileNotFoundError\\
	$\neg$(\textit{eom} $\in \{"ip", "dip", "ea", "dea", "exc"\}$)  & 
	$\Rightarrow$ ValueError\\
	$\neg$(\textit{orthog} $\in \{``symmetric", ``asymmetric"\}$) & 
	$\Rightarrow$ ValueError\\
	$\neg$(\textit{tol} $\in \mathbb{R}$) & $\Rightarrow$ TypeError\\
	$\neg$(\textit{tol} $> 0$) & $\Rightarrow$ ValueError\\
\end{longtable*}

\noindent ParsedParams.\textit{N}:
\begin{itemize}
	%	\item transition: \wss{if appropriate} 
	\item output: \textit{out} := \textit{N} 
	\item exception: None 
\end{itemize}

\noindent ParsedParams.\textit{tol}:
\begin{itemize}
	%	\item transition: \wss{if appropriate} 
	\item output: \textit{out} := \textit{tol} 
	\item exception: None 
\end{itemize}

\noindent ParsedParams.\textit{orthog}:
\begin{itemize}
	%	\item transition: \wss{if appropriate} 
	\item output: \textit{out} := \textit{orthog} 
	\item exception: None 
\end{itemize}

\noindent ParsedParams.\textit{eom}:
\begin{itemize}
	%	\item transition: \wss{if appropriate} 
	\item output: \textit{out} := \textit{eom} 
	\item exception: None 
\end{itemize}

\noindent ParsedParams.one\_int\_file:
\begin{itemize}
	%	\item transition: \wss{if appropriate} 
	\item output: \textit{out} := \textit{one\_int\_file} 
	\item exception: None 
\end{itemize}

\noindent ParsedParams.two\_int\_file:
\begin{itemize}
	%	\item transition: \wss{if appropriate} 
	\item output: \textit{out} := \textit{two\_int\_file} 
	\item exception: None 
\end{itemize}

\noindent ParsedParams.dm1\_file:
\begin{itemize}
	%	\item transition: \wss{if appropriate} 
	\item output: \textit{out} := \textit{two\_int\_file} 
	\item exception: None 
\end{itemize}

\noindent ParsedParams.dm2\_file:
\begin{itemize}
	%	\item transition: \wss{if appropriate} 
	\item output: \textit{out} := \textit{two\_int\_file} 
	\item exception: None 
\end{itemize}

%\noindent load\_integrals(\textit{one\_int\_file}, \textit{two\_int\_file}):
%\begin{itemize}
%%	\item transition: 
%%	The state of the module Integrals is modified following the procedure 
%%bellow:\\
%%	Read the arrays inside the binary files \textit{one\_int\_file} and 
%%\textit{two\_int\_file}\\
%%	Pass the loaded arrays (\textit{one\_int} and \textit{two\_int}) to the 
%%Integrals method constructor.
%	\item output: \textit{out} := \textit{one\_int}, \textit{two\_int}
%	\item exception: None 
%\end{itemize}
%
%\noindent load\_rdms(\textit{dm1\_file}, \textit{dm2\_file}):
%\begin{itemize}
%%	\item transition: 
%	\item output: \textit{out} := \textit{dm1\_int}, \textit{dm1\_int}
%	\item exception: None 
%\end{itemize}

\subsubsection{Local Functions}

%\wss{As appropriate} \wss{These functions are for the purpose of specification.
%	They are not necessarily something that is going to be implemented
%	explicitly.  Even if they are implemented, they are not exported; they only
%	have local scope.}
None

\newpage

\section{MIS of Integrlas Module} \label{mIntegrals}

\subsection{Template Module}

Integrals

\subsection{Uses}
input (\ref{mInput})

\subsection{Syntax}

\subsubsection{Exported Constants}

\subsubsection{Exported Access Programs}

\begin{center}
	\begin{tabular}{p{2.7cm} p{4cm} p{4cm} p{2cm}}
		\hline
		\textbf{Name} & \textbf{In} & \textbf{Out} & \textbf{Exceptions} \\
		\hline
		new Integrlas& str, str& Integrlas& -\\
 		\textbf{h}& -& seq($m,m:\mathbb{R}$)& -\\
 		\textbf{v}& -& seq($m,m,m,m:\mathbb{R}$)& -\\
		\textit{nspino} & -& $\mathbb{Z}$& -\\
		\hline
	\end{tabular}
\end{center}

\subsection{Semantics}

\subsubsection{State Variables}

%\wss{Not all modules will have state variables.  State variables give the 
%module
%	a memory.}
\textbf{h}: seq($m,m:\mathbb{R}$)\\
\textbf{v}: seq($m,m,m,m:\mathbb{R}$)\\
\textit{nspino}: $\mathbb{Z}$

\subsubsection{Environment Variables}

%\wss{This section is not necessary for all modules.  Its purpose is to capture
%	when the module has external interaction with the environment, such as for a
%	device driver, screen interface, keyboard, file, etc.}
intfile1: binary file in NumPy .npy format.\\
intfile2: binary file in NumPy .npy format.

\subsubsection{Assumptions}

%\wss{Try to minimize assumptions and anticipate programmer errors via
%	exceptions, but for practical purposes assumptions are sometimes 
%	appropriate.}
The constructor of Integrals will be called before any state variable is 
invoked.

\subsubsection{Access Routine Semantics}

\noindent new Integrals(\textit{one\_int\_file}, \textit{two\_int\_file}):
\begin{itemize}
	\item transition: Call load\_integrals(\textit{one\_int\_file}, 
	\textit{two\_int\_file})
	\item output: \textit{out} := self 
	\item exception: None 
\end{itemize}

\noindent Integrals.h:
\begin{itemize}
	%	\item transition: \wss{if appropriate} 
	\item output: \textit{out} := \textbf{h} 
	\item exception: None 
\end{itemize}

\noindent Integrals.v:
\begin{itemize}
	%	\item transition: \wss{if appropriate} 
	\item output: \textit{out} := \textbf{v} 
	\item exception: None 
\end{itemize}

\noindent Integrals.nspino:
\begin{itemize}
	%	\item transition: \wss{if appropriate} 
	\item output: \textit{out} := \textit{nspino} 
	\item exception: None 
\end{itemize}

%\wss{A module without environment variables or state variables is unlikely to
%	have a state transition.  In this case a state transition can only occur if
%	the module is changing the state of another module.}
%
%\wss{Modules rarely have both a transition and an output.  In most cases you
%	will have one or the other.}

\subsubsection{Local Functions}
%verify\_integrals& one\_int, two\_int& - & TypeError, ValueError \\

%\wss{As appropriate} \wss{These functions are for the purpose of specification.
%	They are not necessarily something that is going to be implemented
%	explicitly.  Even if they are implemented, they are not exported; they only
%	have local scope.}
\noindent load\_integrals(\textit{one\_int\_file}, \textit{two\_int\_file}):
\begin{itemize}
	\item transition: \\
	Read the binary files \textit{one\_int\_file} and 
	\textit{two\_int\_file} \\
	verify\_integrals()\\
	If no exception is raised, assigne the state variables \textbf{h} and 
	\textbf{v}	
%	\item output: None
	\item exception: exc := FileNotFoundError
\end{itemize}

\noindent verify\_integrals():
\begin{itemize}
%	\item transition: None
	\item output: \textit{out} := None
	\item exception: exc := 
\end{itemize}
\noindent \begin{longtable*}[l]{l l}
	$\neg$(\textbf{h}$\in$ sequence of $\mathbb{R}$) & $\Rightarrow$ TypeError\\
	$\neg$(\textbf{v}$\in$ sequence of $\mathbb{R}$) & $\Rightarrow$ 
	TypeError\\
	\textbf{h} is not a bidimensional arrray  & $\Rightarrow$ ValueError\\
	\textbf{v} is not a 4 dimensional array& $\Rightarrow$ ValueError\\
	$\neg$($|\bold{h}[0]| = |\bold{v}[0]|$) & $\Rightarrow$ 
	ValueError\\
	$\neg$($h_{ij} = h_{ji}$) & $\Rightarrow$ 
	ValueError\\
	$\neg$(($v_{ijkl} = v_{jilk}$) $\land$ ($v_{ijkl} = v_{klij}$)) & 
	$\Rightarrow$ 
	ValueError\\
	$\neg$(($v_{ijkl} = -v_{jikl}$) $\land$ ($v_{ijkl} = -v_{ijlk}$)) & 
	$\Rightarrow$ 
	ValueError\\
\end{longtable*}

\newpage

\section{MIS of RDMs Module} \label{mRDMs}

\subsection{Template Module}

WfnRDMs

\subsection{Uses}
input (\ref{mInput})

\subsection{Syntax}

\subsubsection{Exported Constants}

\subsubsection{Exported Access Programs}

\begin{center}
	\begin{tabular}{p{3cm} p{4cm} p{4cm} p{2cm}}
		\hline
		\textbf{Name} & \textbf{In} & \textbf{Out} & \textbf{Exceptions} \\
		\hline
		new WfnRDMs& $\mathbb{Z} \lor (n1, n2:\mathbb{Z})$, str, 
		str& WfnRDMs& -\\
		$\boldsymbol{\gamma}$& -& seq($m,m:\mathbb{R}$)& -\\
		$\boldsymbol{\Gamma}$& -& seq($m,m,m,m:\mathbb{R}$)& -\\
		\textit{N} & -& $(n1, n2:\mathbb{Z})$& -\\
		\textit{nspino} & -& $\mathbb{Z}$& -\\
%		\textit{hf\_rdms}& -& bool& -\\
		\hline
	\end{tabular}
\end{center}

\subsection{Semantics}

\subsubsection{State Variables}

%\wss{Not all modules will have state variables.  State variables give the 
%module
%	a memory.}
\textit{N}: $(n1, n2:\mathbb{Z})$\\
\textit{nspino}: $\mathbb{Z}$\\
%\textit{nso}: $\mathbb{Z}$\\
$\boldsymbol{\gamma}$: seq($m,m:\mathbb{R}$), where 0 $\le \gamma_{ij} \le$ 1\\
$\boldsymbol{\Gamma}$: seq($m,m,m,m:\mathbb{R}$), where 0 $\le \Gamma_{ijkl} 
\le$ 1\\
%\textit{hf\_rdms}: bool

\subsubsection{Environment Variables}

%\wss{This section is not necessary for all modules.  Its purpose is to capture
%	when the module has external interaction with the environment, such as for a
%	device driver, screen interface, keyboard, file, etc.}
file1: binary file in NumPy .npy format.\\
file2: binary file in NumPy .npy format.

\subsubsection{Assumptions}

%\wss{Try to minimize assumptions and anticipate programmer errors via
%	exceptions, but for practical purposes assumptions are sometimes 
%	appropriate.}
The constructor of WfnRDMs will be called before invoking any state variable.

\subsubsection{Access Routine Semantics}

\noindent new WfnRDMs(n1, \textit{dm1\_file}, \textit{dm2\_file}):
\begin{itemize}
	\item transition: \\
	\textit{N}:= n1\\
	Call assign\_rdms(\textit{dm1\_file}, \textit{dm2\_file})
	\item output: \textit{out} := self 
	\item exception: None 
\end{itemize}

\noindent WfnRDMs.dm1:
\begin{itemize}
	%	\item transition: \wss{if appropriate} 
	\item output: \textit{out} := $\boldsymbol{\gamma}$ 
	\item exception: None 
\end{itemize}

\noindent WfnRDMs.dm2:
\begin{itemize}
	%	\item transition: \wss{if appropriate} 
	\item output: \textit{out} := $\boldsymbol{\Gamma}$ 
	\item exception: None 
\end{itemize}

\noindent WfnRDMs.N:
\begin{itemize}
	%	\item transition: \wss{if appropriate} 
	\item output: \textit{out} := \textit{N} 
	\item exception: None 
\end{itemize}

\noindent WfnRDMs.nspino:
\begin{itemize}
	%	\item transition: \wss{if appropriate} 
	\item output: \textit{out} := \textit{nspino} 
	\item exception: None 
\end{itemize}

%\noindent WfnRDMs.hf\_rdms:
%\begin{itemize}
%	%	\item transition: \wss{if appropriate} 
%	\item output: \textit{out} := \textit{hf\_rdms}
%	\item exception: None 
%\end{itemize}

%\wss{A module without environment variables or state variables is unlikely to
%	have a state transition.  In this case a state transition can only occur if
%	the module is changing the state of another module.}
%
%\wss{Modules rarely have both a transition and an output.  In most cases you
%	will have one or the other.}

\subsubsection{Local Functions}
%verify\_integrals& one\_int, two\_int& - & TypeError, ValueError \\

%\wss{As appropriate} \wss{These functions are for the purpose of specification.
%	They are not necessarily something that is going to be implemented
%	explicitly.  Even if they are implemented, they are not exported; they only
%	have local scope.}
\noindent assign\_rdms(\textit{dm1\_file},\textit{dm2\_file}):
\begin{itemize}
	\item transition: Read the binary files \textit{dm1\_file} and 
	\textit{dm2\_file}.\\
	verify\_rdms()\\
	If no exception is raised, assign the state variables $\boldsymbol{\gamma}$ 
	and $\boldsymbol{\Gamma}$	
	%	\item output: None
	\item exception: exc := FileNotFoundError
\end{itemize}

\noindent verify\_rdms():
\begin{itemize}
	%	\item transition: None
	\item output: \textit{out} := None
	\item exception: exc := 
\end{itemize}
\noindent \begin{longtable*}[l]{l l}
	$\neg(\boldsymbol{\gamma}\in$ sequence of $\mathbb{R})$ & $\Rightarrow$ 
	TypeError\\
	$\neg(\boldsymbol{\Gamma}\in$ sequence of $\mathbb{R})$ & $\Rightarrow$ 
	TypeError\\
	$\boldsymbol{\gamma}$ is not a bidimensional arrray  & $\Rightarrow$ 
	ValueError\\
	$\boldsymbol{\Gamma}$ is not a 4 dimensional array& $\Rightarrow$ 
	ValueError\\
	$\neg$($\gamma_{ij} = \gamma_{ji}$) & $\Rightarrow$ 
	ValueError\\
	$\neg$($\Gamma_{ijkl} = \Gamma_{jilk}$)$\lor \neg(\Gamma_{ijkl} = 
	\Gamma_{klij})$ & 
	$\Rightarrow$ 
	ValueError\\
	$\neg$($\Gamma_{ijkl} = -\Gamma_{jikl}$)$\lor \neg(\Gamma_{ijkl} = 
	-\Gamma_{ijlk})$ & 
	$\Rightarrow$ 
	ValueError\\
	Tr$(\boldsymbol{\gamma}) \neq N$ & $\Rightarrow$ 
	ValueError\\
	Tr$(\boldsymbol{\Gamma}) \neq N(N-1)$ & $\Rightarrow$ 
	ValueError\\
\end{longtable*}

\newpage

\section{MIS of EOM Base Module} \label{mEOM}

\subsection{Interface Module}

EOMBase

\subsection{Uses}
None

\subsection{Syntax}

\subsubsection{Exported Constants}
None

\subsubsection{Exported Access Programs}

\begin{center}
	\begin{tabular}{p{2.3cm} p{4cm} p{4cm} p{2cm}}
		\hline
		\textbf{Name} & \textbf{In} & \textbf{Out} & \textbf{Exceptions} \\
		\hline
		%		new EOMBase & one\_int: 2D NumPy.ndarray, two\_int: 4D 
		%NumPy.ndarray, 
		%		dm1: 2D NumPy.ndarray, dm2: 4D NumPy.ndarray& EOMBase & 
		%		NotImplementedError \\		
		\textit{neigs} & -& $\mathbb{Z}$& NotImplementedError\\
		compute\_tdm& seq(k,k:$\mathbb{R}$)& seq(k,m,m:$\mathbb{R}$)& 
		NotImplementedError\\
		\textit{lhs} & -& seq$(k,k:\mathbb{R})$& -\\
		\textit{rhs} & -& seq$(k,k:\mathbb{R})$& -\\
		\textit{nspino}& -& $\mathbb{Z}$& -\\
		\textbf{h}& -& seq$(m,m:\mathbb{R})$& -\\
		\textbf{v}& -& seq$(m,m,m,m:\mathbb{R})$& -\\
		$\boldsymbol{\gamma}$& -& seq$(m,m:\mathbb{R})$& -\\
		$\boldsymbol{\Gamma}$& -& seq$(m,m,m,m:\mathbb{R})$& -\\
		% 		\_compute\_lhs& -& lhs: $f(one\_int, two\_int, dm1, dm2)$& 
		% 		NotImplementedError\\
		%		\_compute\_rhs& -& rhs: $f(dm1, dm2)$& NotImplementedError\\
		\hline
	\end{tabular}
\end{center}

\subsection{Semantics}

\subsubsection{State Variables}
\textit{nspino}: $\mathbb{Z}$\\
\textbf{h}: seq$(m,m:\mathbb{R})$\\
\textbf{v}: seq$(m,m,m,m:\mathbb{R})$\\
$\boldsymbol{\gamma}$: seq$(m,m:\mathbb{R})$\\
$\boldsymbol{\Gamma}$: seq$(m,m,m,m:\mathbb{R})$\\
\textit{lhs}: \_compute\_lhs()\\
%seq$(k,k:\mathbb{R})$\\
\textit{rhs}: \_compute\_rhs()
%seq$(k,k:\mathbb{R})$

%\subsubsection{Environment Variables}
%
%%\wss{This section is not necessary for all modules.  Its purpose is to capture
%%	when the module has external interaction with the environment, such as for a
%%	device driver, screen interface, keyboard, file, etc.}
%None
%
\subsubsection{Assumptions}

%\wss{Try to minimize assumptions and anticipate programmer errors via
%	exceptions, but for practical purposes assumptions are sometimes 
%	appropriate.}
The EOMBase module can't be instantiated, it is inherited by EOMIP, EOMEA, 
EOMExc, EOMDIP and EOMDEA.
%
%\subsubsection{Access Routine Semantics}
%
%%\wss{A module without environment variables or state variables is unlikely to
%%	have a state transition.  In this case a state transition can only occur if
%%	the module is changing the state of another module.}
%%
%%\wss{Modules rarely have both a transition and an output.  In most cases you
%%	will have one or the other.}
%
\subsubsection{Local Functions}

%\wss{As appropriate} \wss{These functions are for the purpose of 
%specification.
%	They are not necessarily something that is going to be implemented
%	explicitly.  Even if they are implemented, they are not exported; they only
%	have local scope.}
\noindent \_compute\_lhs():
\begin{itemize}
	%	\item transition: \wss{if appropriate} 
%	\item output: out: None 
	\item exception: NotImplementedError 
\end{itemize}

\noindent\_compute\_rhs():
\begin{itemize}
	%	\item transition: \wss{if appropriate} 
%	\item output: out: None 
	\item exception: NotImplementedError 
\end{itemize}

\subsubsection{Considerations}
EOMBase is an abstract class (ABC) defining an interface for the 
different EOM methods (Subsections (\ref{mEOMIP}), (\ref{mEOMEA}), 
(\ref{mEOMExc}), (\ref{mEOMDIP}) and (\ref{mEOMDEA})). Each state variable has a 
corresponding access program. Only the methods \textit{neigs}, compute\_tdm, 
\_compute\_lhs and \_compute\_rhs are abstract.

\newpage

\section{MIS of EOM IP Module} \label{mEOMIP}

\subsection{Template Module}

EOMIP inherits EOMBase

\subsection{Uses}
EOMBase (\ref{mEOM}), Integrals (\ref{mIntegrals}), WfnRDMs (\ref{mRDMs})

\subsection{Syntax}

\subsubsection{Exported Constants}
None

\subsubsection{Exported Access Programs}

\begin{center}
	\begin{tabular}{p{2.3cm} p{4cm} p{4cm} p{2cm}}
		\hline
		\textbf{Name} & \textbf{In} & \textbf{Out} & \textbf{Exceptions} \\
		\hline
		new EOMIP & seq$(m,m:\mathbb{R})$, seq$(m,m,m,m:\mathbb{R})$, 
		seq$(m,m:\mathbb{R})$, seq$(m,m,m,m:\mathbb{R})$& EOMIP & - \\
%		\textit{neigs} & -& $\mathbb{Z}$& -\\
%		compute\_tdm& seq(m,m:$\mathbb{R}$)& seq(m,m,m:$\mathbb{R}$)& 
%		-\\
%		\textit{nspino}& -& $\mathbb{Z}$& -\\
%		\textbf{h}& -& seq$(m,m:\mathbb{R})$& -\\
%		\textbf{v}& -& seq$(m,m,m,m:\mathbb{R})$& -\\
%		$\boldsymbol{\gamma}$& -& seq$(m,m:\mathbb{R})$& -\\
%		$\boldsymbol{\Gamma}$& -& seq$(m,m,m,m:\mathbb{R})$& -\\
%		\textit{lhs} & -& seq$(m,m:\mathbb{R})$& -\\
%		\textit{rhs} & -& seq$(m,m:\mathbb{R})$& -\\
		\hline
	\end{tabular}
\end{center}

\subsection{Semantics}

\subsubsection{State Variables}
\textit{nspino}: $\mathbb{Z}$\\
\textbf{h}: seq$(m,m:\mathbb{R})$\\
\textbf{v}: seq$(m,m,m,m:\mathbb{R})$\\
$\boldsymbol{\gamma}$: seq$(m,m:\mathbb{R})$\\
$\boldsymbol{\Gamma}$: seq$(m,m,m,m:\mathbb{R})$\\
\textit{lhs}: \_compute\_lhs()\\
%seq$(m,m:\mathbb{R})$\\
\textit{rhs}: \_compute\_rhs()
%seq$(m,m:\mathbb{R})$

\subsubsection{Environment Variables}

%\wss{This section is not necessary for all modules.  Its purpose is to capture
%	when the module has external interaction with the environment, such as for a
%	device driver, screen interface, keyboard, file, etc.}
None

\subsubsection{Assumptions}

%\wss{Try to minimize assumptions and anticipate programmer errors via
%	exceptions, but for practical purposes assumptions are sometimes 
%	appropriate.}
The EOMIP constructor is called before any other access program in the class.

\subsubsection{Access Routine Semantics}

\noindent new EOMIP(h,v,dm1,dm2):
\begin{itemize}
	\item transition: \textbf{h}, \textbf{v}, $\boldsymbol{\gamma}$, 
	$\boldsymbol{\Gamma}$ := h,v,dm1,dm2,\\ \textit{lhs}:= \_compute\_lhs()\\ 
	\textit{rhs}:= \_compute\_rhs()\\
	\textit{nspino} := $|\boldsymbol{h}[0]|$\\
	\item output: \textit{out} := self 
	\item exception: None
\end{itemize}

\noindent neigs():
\begin{itemize}
	%	\item transition: \wss{if appropriate} 
	\item output: \textit{out} := $|\boldsymbol{h}[0]|$
	\item exception: None
\end{itemize}

\noindent compute\_tdm(\textbf{c}):
\begin{itemize}
	%	\item transition: \wss{if appropriate} 
	\item output: \textit{out} := $\sum_{n}\gamma_{mn} c_n, \{n:\mathbb{Z}|0 
	\le n < nspino\}$
	\item exception: None
\end{itemize}

\noindent EOMIP.nspino:
\begin{itemize}
	%	\item transition: \wss{if appropriate} 
	\item output: \textit{out} := \textit{nspino}
	\item exception: None
\end{itemize}

\noindent EOMIP.h:
\begin{itemize}
	%	\item transition: \wss{if appropriate} 
	\item output: \textit{out} := \textbf{h}
	\item exception: None
\end{itemize}

\noindent EOMIP.v:
\begin{itemize}
	%	\item transition: \wss{if appropriate} 
	\item output: \textit{out} := \textbf{v}
	\item exception: None
\end{itemize}

\noindent EOMIP.dm1:
\begin{itemize}
	%	\item transition: \wss{if appropriate} 
	\item output: \textit{out} := $\boldsymbol{\gamma}$
	\item exception: None
\end{itemize}

\noindent EOMIP.dm2:
\begin{itemize}
	%	\item transition: \wss{if appropriate} 
	\item output: \textit{out} := $\boldsymbol{\Gamma}$
	\item exception: None
\end{itemize}

\noindent EOMIP.lhs:
\begin{itemize}
	%	\item transition: \wss{if appropriate} 
	\item output: \textit{out} := \textit{lhs} $\in$ seq$(m,m:\mathbb{R})$
	\item exception: ValueError
\end{itemize}

\noindent EOMIP.rhs:
\begin{itemize}
	%	\item transition: \wss{if appropriate} 
	\item output: \textit{out} := \textit{rhs} $\in$ seq$(m,m:\mathbb{R})$
	\item exception: ValueError
\end{itemize}

%\wss{A module without environment variables or state variables is unlikely to
%	have a state transition.  In this case a state transition can only occur if
%	the module is changing the state of another module.}
%
%\wss{Modules rarely have both a transition and an output.  In most cases you
%	will have one or the other.}

\subsubsection{Local Functions}

%\wss{As appropriate} \wss{These functions are for the purpose of specification.
%	They are not necessarily something that is going to be implemented
%	explicitly.  Even if they are implemented, they are not exported; they only
%	have local scope.}
\noindent \_compute\_lhs():
\begin{itemize}
	%	\item transition: \wss{if appropriate} 
	\item output: \textit{out} :=\\
	$-\boldmath{h}\boldsymbol{\gamma}$ $+ 0.5\sum_{qrs}$v$_{qnrs} \Gamma_{mqrs}$
	\item exception: None 
\end{itemize}

\noindent\_compute\_rhs():
\begin{itemize}
	%	\item transition: \wss{if appropriate} 
	\item output: \textit{out} := $\boldsymbol{\gamma}$
	\item exception: None 
\end{itemize}

\newpage

\section{MIS of EOM EA Module} \label{mEOMEA}

\subsection{Template Module}

EOMEA inherits EOMBase

\subsection{Uses}
EOMBase (\ref{mEOM}), Integrals (\ref{mIntegrals}), WfnRDMs (\ref{mRDMs})

\subsection{Syntax}

\subsubsection{Exported Constants}
None

\subsubsection{Exported Access Programs}

\begin{center}
	\begin{tabular}{p{2.3cm} p{4cm} p{4cm} p{2cm}}
		\hline
		\textbf{Name} & \textbf{In} & \textbf{Out} & \textbf{Exceptions} \\
		\hline
		new EOMEA & seq$(m,m:\mathbb{R})$, seq$(m,m,m,m:\mathbb{R})$, 
		seq$(m,m:\mathbb{R})$, seq$(m,m,m,m:\mathbb{R})$& EOMEA & - \\
		\hline
	\end{tabular}
\end{center}

\subsection{Semantics}

\subsubsection{State Variables}
\textit{nspino}: $\mathbb{Z}$\\
\textbf{h}: seq$(m,m:\mathbb{R})$\\
\textbf{v}: seq$(m,m,m,m:\mathbb{R})$\\
$\boldsymbol{\gamma}$: seq$(m,m:\mathbb{R})$\\
$\boldsymbol{\Gamma}$: seq$(m,m,m,m:\mathbb{R})$\\
\textit{lhs}: \_compute\_lhs()\\
%seq$(m,m:\mathbb{R})$\\
\textit{rhs}: \_compute\_rhs()
%seq$(m,m:\mathbb{R})$

\subsubsection{Environment Variables}

%\wss{This section is not necessary for all modules.  Its purpose is to capture
%	when the module has external interaction with the environment, such as for a
%	device driver, screen interface, keyboard, file, etc.}
None

\subsubsection{Assumptions}

%\wss{Try to minimize assumptions and anticipate programmer errors via
%	exceptions, but for practical purposes assumptions are sometimes 
%	appropriate.}
The EOMEA constructor is called before any other access program in that class.

\subsubsection{Access Routine Semantics}

\noindent new EOMEA(h,v,dm1,dm2):
\begin{itemize}
	\item transition: \textbf{h}, \textbf{v}, $\boldsymbol{\gamma}$, 
	$\boldsymbol{\Gamma}$ := h,v,dm1,dm2,\\ \textit{lhs}:= \_compute\_lhs(),\\ 
	\textit{rhs}:= \_compute\_rhs()\\
	\textit{nspino} := $|\boldsymbol{h}[0]|$\\
	\item output: \textit{out} := self 
	\item exception: None
\end{itemize}

\noindent neigs():
\begin{itemize}
	%	\item transition: \wss{if appropriate} 
	\item output: \textit{out} := $|\boldsymbol{h}[0]|$
	\item exception: None
\end{itemize}

\noindent compute\_tdm(\textbf{c}):
\begin{itemize}
	%	\item transition: \wss{if appropriate} 
	\item output: \textit{out} := $\sum_{n}(\delta_{mn} - \gamma_{mn}) c_n, 
	\{n:\mathbb{Z}|0 \le n < nspino\}$
	\item exception: None
\end{itemize}

\noindent EOMEA.nspino:
\begin{itemize}
	%	\item transition: \wss{if appropriate} 
	\item output: \textit{out} := \textit{nspino}
	\item exception: None
\end{itemize}

\noindent EOMEA.h:
\begin{itemize}
	%	\item transition: \wss{if appropriate} 
	\item output: \textit{out} := \textbf{h}
	\item exception: None
\end{itemize}

\noindent EOMEA.v:
\begin{itemize}
	%	\item transition: \wss{if appropriate} 
	\item output: \textit{out} := \textbf{v}
	\item exception: None
\end{itemize}

\noindent EOMEA.dm1:
\begin{itemize}
	%	\item transition: \wss{if appropriate} 
	\item output: \textit{out} := $\boldsymbol{\gamma}$
	\item exception: None
\end{itemize}

\noindent EOMEA.dm2:
\begin{itemize}
	%	\item transition: \wss{if appropriate} 
	\item output: \textit{out} := $\boldsymbol{\Gamma}$
	\item exception: None
\end{itemize}

\noindent EOMEA.lhs:
\begin{itemize}
	%	\item transition: \wss{if appropriate} 
	\item output: \textit{out} := \textit{lhs} $\in$ seq$(m,m:\mathbb{R})$
	\item exception: ValueError
\end{itemize}

\noindent EOMEA.rhs:
\begin{itemize}
	%	\item transition: \wss{if appropriate} 
	\item output: \textit{out} := \textit{rhs} $\in$ seq$(m,m:\mathbb{R})$
	\item exception: ValueError
\end{itemize}

%\wss{A module without environment variables or state variables is unlikely to
%	have a state transition.  In this case a state transition can only occur if
%	the module is changing the state of another module.}
%
%\wss{Modules rarely have both a transition and an output.  In most cases you
%	will have one or the other.}

\subsubsection{Local Functions}

%\wss{As appropriate} \wss{These functions are for the purpose of specification.
%	They are not necessarily something that is going to be implemented
%	explicitly.  Even if they are implemented, they are not exported; they only
%	have local scope.}
\noindent \_compute\_lhs():
\begin{itemize}
	%	\item transition: \wss{if appropriate} 
	\item output: \textit{out} :=\\
	\textbf{h} - \textbf{h}$\boldsymbol{\gamma}
	+\sum_{ps}$v$_{mpns} \gamma_{ps} 
	+ 0.5\sum_{pqs}$v$_{pqns}\Gamma_{pqsm}$
	\item exception: None 
\end{itemize}

\noindent\_compute\_rhs():
\begin{itemize}
	%	\item transition: \wss{if appropriate} 
	\item output: \textit{out} := $\boldmath{I} - \boldsymbol{\gamma}$, where 
	\textbf{I} represents the identity matrix
	\item exception: None 
\end{itemize}

\newpage

\section{MIS of EOM Excitation Module} \label{mEOMExc}

\subsection{Template Module}

EOMExc inherits EOMBase

\subsection{Uses}
EOMBase (\ref{mEOM}), Integrals (\ref{mIntegrals}), WfnRDMs (\ref{mRDMs})

\subsection{Syntax}

\subsubsection{Exported Constants}
None

\subsubsection{Exported Access Programs}

\begin{center}
	\begin{tabular}{p{2.3cm} p{4cm} p{4cm} p{2cm}}
		\hline
		\textbf{Name} & \textbf{In} & \textbf{Out} & \textbf{Exceptions} \\
		\hline
		new EOMExc & seq$(m,m:\mathbb{R})$, seq$(m,m,m,m:\mathbb{R})$, 
		seq$(m,m:\mathbb{R})$, seq$(m,m,m,m:\mathbb{R})$& EOMExc & - \\
%		\textit{nspino}& -& $\mathbb{Z}$& -\\
%		\textit{neigs} & -& $\mathbb{Z}$& -\\
%		\textbf{h}& -& seq$(m,m:\mathbb{R})$& -\\
%		\textbf{v}& -& seq$(m,m,m,m:\mathbb{R})$& -\\
%		$\boldsymbol{\gamma}$& -& seq$(m,m:\mathbb{R})$& -\\
%		$\boldsymbol{\Gamma}$& -& seq$(m,m,m,m:\mathbb{R})$& -\\
%		\textit{lhs} & -& seq$(m^2,m^2:\mathbb{R})$& -\\
%		\textit{rhs} & -& seq$(m^2,m^2:\mathbb{R})$& -\\
%		compute\_tdm& seq($m^2,m^2:\mathbb{R}$)& seq($m^2,m,m:\mathbb{R}$)& 
%		-\\
		\hline
	\end{tabular}
\end{center}

\subsection{Semantics}

\subsubsection{State Variables}
\textit{nspino}: $\mathbb{Z}$\\
\textbf{h}: seq$(m,m:\mathbb{R})$\\
\textbf{v}: seq$(m,m,m,m:\mathbb{R})$\\
$\boldsymbol{\gamma}$: seq$(m,m:\mathbb{R})$\\
$\boldsymbol{\Gamma}$: seq$(m,m,m,m:\mathbb{R})$\\
\textit{lhs}: \_compute\_lhs()\\
\textit{rhs}: \_compute\_rhs()

\subsubsection{Environment Variables}

%\wss{This section is not necessary for all modules.  Its purpose is to capture
%	when the module has external interaction with the environment, such as for a
%	device driver, screen interface, keyboard, file, etc.}
None

\subsubsection{Assumptions}

%\wss{Try to minimize assumptions and anticipate programmer errors via
%	exceptions, but for practical purposes assumptions are sometimes 
%	appropriate.}
The EOMExc constructor is called before any other access program in that class.

\subsubsection{Access Routine Semantics}

\noindent new EOMExc(h,v,dm1,dm2):
\begin{itemize}
	\item transition: \textbf{h}, \textbf{v}, $\boldsymbol{\gamma}$, 
	$\boldsymbol{\Gamma}$ := h,v,dm1,dm2,\\ \textit{lhs}:= \_compute\_lhs(),\\ 
	\textit{rhs}:= \_compute\_rhs()\\
	\textit{nspino} := $|\boldsymbol{h}[0]|$\\
	\textit{neigs} := $|\boldsymbol{h}[0]|$
	\item output: \textit{out} := self 
	\item exception: None
\end{itemize}

neigs():
\begin{itemize}
	%	\item transition: \wss{if appropriate} 
	\item output: \textit{out} := $|\boldsymbol{h}[0]|^2 \in \mathbb{Z}$
	\item exception: None
\end{itemize}

\noindent compute\_tdm(\textbf{c}):
\begin{itemize}
	%	\item transition: \wss{if appropriate} 
	\item output: \textit{out} := $\sum_{ij}(\delta_{li}\gamma_{kj} - 
	\Gamma_{kijl}) c_{ij}, \{(i,j)|(i\in [0..\text{nspino}-1]) \land (j\in 
	[0..\text{nspino}-1])\}$
	\item exception: None
\end{itemize}

\noindent EOMExc.nspino:
\begin{itemize}
	%	\item transition: \wss{if appropriate} 
	\item output: \textit{out} := \textit{nspino}
	\item exception: None
\end{itemize}

\noindent EOMExc.h:
\begin{itemize}
	%	\item transition: \wss{if appropriate} 
	\item output: \textit{out} := \textbf{h}
	\item exception: None
\end{itemize}

\noindent EOMExc.v:
\begin{itemize}
	%	\item transition: \wss{if appropriate} 
	\item output: \textit{out} := \textbf{v}
	\item exception: None
\end{itemize}

\noindent EOMExc.dm1:
\begin{itemize}
	%	\item transition: \wss{if appropriate} 
	\item output: \textit{out} := $\boldsymbol{\gamma}$
	\item exception: None
\end{itemize}

\noindent EOMExc.dm2:
\begin{itemize}
	%	\item transition: \wss{if appropriate} 
	\item output: \textit{out} := $\boldsymbol{\Gamma}$
	\item exception: None
\end{itemize}

\noindent EOMExc.lhs:
\begin{itemize}
	%	\item transition: \wss{if appropriate} 
	\item output: \textit{out} := \textit{lhs} $\in$ seq$(m^2,m^2:\mathbb{R})$
	\item exception: ValueError
\end{itemize}

\noindent EOMExc.rhs:
\begin{itemize}
	%	\item transition: \wss{if appropriate} 
	\item output: \textit{out} := \textit{rhs} $\in$ seq$(m^2,m^2:\mathbb{R})$
	\item exception: ValueError
\end{itemize}

%\wss{A module without environment variables or state variables is unlikely to
%	have a state transition.  In this case a state transition can only occur if
%	the module is changing the state of another module.}
%
%\wss{Modules rarely have both a transition and an output.  In most cases you
%	will have one or the other.}

\subsubsection{Local Functions}

%\wss{As appropriate} \wss{These functions are for the purpose of specification.
%	They are not necessarily something that is going to be implemented
%	explicitly.  Even if they are implemented, they are not exported; they only
%	have local scope.}
\noindent \_compute\_lhs():
\begin{itemize}
	%	\item transition: \wss{if appropriate} 
	\item output: \textit{out} := \\
	$h_{li}\gamma_{kj} + h_{jk}\gamma_{il} - \sum_{q} 
	(h_{jq}\delta_{li}\gamma_{kq} + h_{qi}\delta_{jk}\gamma_{ql})\\
	+ \sum_{qs}($v$_{lqis}\Gamma_{kqjs} + $v$_{jqks}\Gamma_{iqls})\\
	+ 0.5\sum_{rs}($v$_{jlrs}\Gamma_{kirs} + 
	\sum_{q}$v$_{qjrs}\delta_{li}\Gamma_{kqrs})\\
	+ 0.5 \sum_{pq}($v$_{pqik}\Gamma_{pqlj} + 
	\sum_{s}$v$_{pqsi}\delta_{jk}\Gamma_{pqls})$
	\item exception: None 
\end{itemize}

\noindent\_compute\_rhs():
\begin{itemize}
	%	\item transition: \wss{if appropriate} 
	\item output: \textit{out} := $\delta_{li}\gamma_{kj} - 
	\boldsymbol{\Gamma}$
	\item exception: None 
\end{itemize}

\newpage

\section{MIS of EOM DIP Module} \label{mEOMDIP}
The MIS of EOM DIP is equivalent to the one for EOM Excitation (Section 
\ref{mEOMExc}), therefore only the semantics of the methods that change will be 
declared.

\subsection{Template Module}

EOMDIP inherits EOMBase

\subsection{Uses}
EOMBase (\ref{mEOM}), Integrals (\ref{mIntegrals}), WfnRDMs (\ref{mRDMs})

\subsection{Access Routine Semantics}
\noindent compute\_tdm(\textbf{c}):
\begin{itemize}
	%	\item transition: \wss{if appropriate} 
	\item output: \textit{out} := $\sum_{ij}\Gamma_{klji} c_{ij}, \{(i,j)|(i\in 
	[0..\text{nspino}-1]) \land (j\in 
	[0..\text{nspino}-1])\}$
	\item exception: None
\end{itemize}

\subsubsection{Local Functions}
%\wss{As appropriate} \wss{These functions are for the purpose of specification.
%	They are not necessarily something that is going to be implemented
%	explicitly.  Even if they are implemented, they are not exported; they only
%	have local scope.}
\noindent \_compute\_lhs():
\begin{itemize}
	%	\item transition: \wss{if appropriate} 
	\item output: \textit{out} := \\
	$2(h_{jk}\delta_{il} -h_{jl} \delta_{ik} 
	+ h_{ik} \gamma_{lj} - h_{il}\gamma_{kj})\\
	+ 2 \sum_{q} h_{jq}( \delta_{ik} \gamma_{lq} - \delta_{il} \gamma_{kq})
	+ \mathbf{v} \\+ 2 \sum_{q}  $v$_{qjkl} \gamma_{qi}
	+ \sum_{r} ($v$_{jilr} \gamma_{kr} - 
	$v$_{jikr} \gamma_{lr})\\
	+ 2 \sum_{qr} ($v$_{iqrk}  \delta_{lj} + $v$_{iqlr} 
	\delta_{kj})\gamma_{qr}\\
	+ 2 \sum_{qr} ($v$_{jqrk}\Gamma_{qlri} + 
	$v$_{jqlr}\Gamma_{qkri})\\
	+ \sum_{qrs} $v$_{qjrs}(\delta_{ki}\Gamma_{qlrs} - 
	\delta_{li}\Gamma_{qkrs})$
	\item exception: None 
\end{itemize}

\noindent\_compute\_rhs():
\begin{itemize}
	%	\item transition: \wss{if appropriate} 
	\item output: \textit{out} := $2\delta_{jk} \gamma_{li} +2\delta_{il} 
	\gamma_{kj}	- 2\delta_{jk} \delta_{il}$
	\item exception: None 
\end{itemize}

\newpage

\section{MIS of EOM DEA Module} \label{mEOMDEA}
The MIS of EOM DEA is equivalent to the one for EOM Excitation (Section 
\ref{mEOMExc}), therefore only the mothods that change are declared.

\subsection{Template Module}

EOMDEA inherits EOMBase

\subsection{Uses}
EOMBase (\ref{mEOM}), Integrals (\ref{mIntegrals}), WfnRDMs (\ref{mRDMs})

\subsection{Access Routine Semantics}
\noindent compute\_tdm(\textbf{c}):
\begin{itemize}
	%	\item transition: \wss{if appropriate} 
	\item output: \textit{out} := \\
	$\sum_{ij}(2\delta_{li}\delta_{kj} + 2\delta_{lj}\gamma_{ik} + 2 
	2\delta_{ki}\gamma_{jl} + \Gamma_{ijlk})c_{ij}, \{(i,j)|(i\in 
	[0..\text{nspino}-1]) \land (j\in 
	[0..\text{nspino}-1])\}$
	\item exception: None
\end{itemize}

\subsubsection{Local Functions}
%\wss{As appropriate} \wss{These functions are for the purpose of specification.
%	They are not necessarily something that is going to be implemented
%	explicitly.  Even if they are implemented, they are not exported; they only
%	have local scope.}
\noindent \_compute\_lhs():
\begin{itemize}
	%	\item transition: \wss{if appropriate} 
	\item output: \textit{out} := \\
	$2(h_{li} \delta_{kj} - h_{ki}\delta_{lj} + h_{ki} \gamma_{jl} - h_{li} 
	\gamma_{jk})\\
	+ 2 \sum_{p} (h_{pi} \delta_{lj}\gamma_{pk} + h_{pj} 
	\delta_{ki}\gamma_{pl})\\
	+$\textbf{v}$+ 2 \sum_{r}$v$_{lkjr} \gamma_{ir} \\
	+ \sum_{q} ($v$_{qlij} \gamma_{qk} - 
	$v$_{qkij} \gamma_{ql})\\
	+ 2 \sum_{qr} ($v$_{qljr}  \delta_{ki} - $v$_{qkjr} 
	\delta_{li})\gamma_{qr}\\
	+ 2 \sum_{qr} ($v$_{qlir}\Gamma_{qjrk} - 
	$v$_{qkir}\Gamma_{qjrl})\\
	+ \sum_{pqr} $v$_{pqjr}(\delta_{li}\Gamma_{pqrk} - 
	\delta_{ki}\Gamma_{pqrl})$
	\item exception: None 
\end{itemize}

\noindent\_compute\_rhs():
\begin{itemize}
	%	\item transition: \wss{if appropriate} 
	\item output: \textit{out} := $2\delta_{li}\delta_{kj} - 
	2\delta_{li}\gamma_{jk}- 2\delta_{kj}\gamma_{il}$
	\item exception: None 
\end{itemize}

\newpage

\section{MIS of Output module} \label{mOutput} 

\subsection{Module}

output

\subsection{Uses}
input (\ref{mInput})

\subsection{Syntax}

\subsubsection{Exported Constants}

\subsubsection{Exported Access Programs}

\begin{center}
	\begin{tabular}{p{2cm} p{4cm} p{4cm} p{2cm}}
		\hline
		\textbf{Name} & \textbf{In} & \textbf{Out} & \textbf{Exceptions} \\
		\hline
		output & fname: str, $\boldsymbol{\Delta E}$:seq(k:$\mathbb{R}$), 
		$\boldsymbol{c}$=seq(k,n:$\mathbb{R}$), 
		$\boldsymbol{\gamma_{n;0k}}$=seq(k,n,n:$\mathbb{R}$)& - & - \\
% 			   & $c_{n;k}$=NumPy.ndarray& & \\
%   	  		   & $\gamma_{n;0k}$=NumPy.ndarray& & \\
		\hline
	\end{tabular}
\end{center}

\subsection{Semantics}

\subsubsection{State Variables}
None

\subsubsection{Environment Variables}
outputFile: A text file

\subsubsection{Assumptions}

%\wss{Try to minimize assumptions and anticipate programmer errors via
%	exceptions, but for practical purposes assumptions are sometimes 
%	appropriate.}

\subsubsection{Access Routine Semantics}

\noindent output(fname,$\boldsymbol{\Delta 
E}$,$\boldsymbol{c}$,$\boldsymbol{\gamma_{n;0k}}$):
\begin{itemize}
	\item transition: Write to fname the input parameters from ParsedParams and 
	the results of the calculations: $\boldsymbol{\Delta 
		E}$, $\boldsymbol{c}$ and $\boldsymbol{\gamma_{n;0k}}$
%	\item output: None 
	\item exception: None
\end{itemize}

%\wss{A module without environment variables or state variables is unlikely to
%	have a state transition.  In this case a state transition can only occur if
%	the module is changing the state of another module.}
%
%\wss{Modules rarely have both a transition and an output.  In most cases you
%	will have one or the other.}

\subsubsection{Local Functions}

%\wss{As appropriate} \wss{These functions are for the purpose of specification.
%	They are not necessarily something that is going to be implemented
%	explicitly.  Even if they are implemented, they are not exported; they only
%	have local scope.}
None

\newpage

\section{MIS of Solver Module} \label{mGEP}

\subsection{Module}

solve

\subsection{Uses}
input (\ref{mInput})

\subsection{Syntax}

\subsubsection{Exported Constants}

\subsubsection{Exported Access Programs}

\begin{center}
	\begin{tabular}{p{2cm} p{4cm} p{4cm} p{2cm}}
		\hline
		\textbf{Name} & \textbf{In} & \textbf{Out} & \textbf{Exceptions} \\
		\hline
		dense & $\textbf{A}$: seq(k,k:$\mathbb{R}$), $\textbf{B}$: 
		seq(k,k:$\mathbb{R}$), 
		tol: $\mathbb{R} >0$, 
		orthog: str $in \{"symm", "asymm"\}$& 
		$\boldsymbol{\Delta E}$:seq(k:$\mathbb{R}$), 
		$\boldsymbol{c}$=seq(k,k:$\mathbb{R}$) & 
		DivideByZero \\
		\hline
	\end{tabular}
\end{center}

\subsection{Semantics}

\subsubsection{State Variables}

%\wss{Not all modules will have state variables.  State variables give the 
%module
%	a memory.}

\subsubsection{Environment Variables}

%\wss{This section is not necessary for all modules.  Its purpose is to capture
%	when the module has external interaction with the environment, such as for a
%	device driver, screen interface, keyboard, file, etc.}

\subsubsection{Assumptions}

%\wss{Try to minimize assumptions and anticipate programmer errors via
%	exceptions, but for practical purposes assumptions are sometimes 
%	appropriate.}

\subsubsection{Access Routine Semantics}

\noindent dense($\textbf{A}$,$\textbf{B}$,tol,orthog):
\begin{itemize}
%	\item transition: solve a generalized eigenvalue problem.
	\item output: \textit{out} := $\boldsymbol{\Delta E}$, $\boldsymbol{c}$ 
	that satifies $\boldsymbol{A}c_i=\Delta E_i \boldsymbol{B} c_i, \{i|0\le i 
	\le k\} $
	\item exception: DivideByZero 
\end{itemize}

%\wss{A module without environment variables or state variables is unlikely to
%	have a state transition.  In this case a state transition can only occur if
%	the module is changing the state of another module.}
%
%\wss{Modules rarely have both a transition and an output.  In most cases you
%	will have one or the other.}

\subsubsection{Local Functions}

%\wss{As appropriate} \wss{These functions are for the purpose of specification.
%	They are not necessarily something that is going to be implemented
%	explicitly.  Even if they are implemented, they are not exported; they only
%	have local scope.}
None

\newpage

\bibliographystyle {plainnat}
\bibliography {../../../refs/References}

\newpage

\section{Appendix} \label{Appendix}

%\wss{Extra information if required}

\end{document}